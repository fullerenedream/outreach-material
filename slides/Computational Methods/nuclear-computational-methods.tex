%\documentclass[14pt]{beamer}
\documentclass[pdf,aspectratio=169]{beamer}
%\documentclass[pdf]{beamer}
%\mode<presentation>{}
%% preamble
\usepackage{svg}
%\usetheme[]{Boadilla}
\usetheme[]{default}
\usepackage{subfig}
\usepackage{graphicx}
\usepackage{ulem}
\usepackage{enumitem}
\usepackage{amsmath,amssymb,amsfonts}
\usepackage{tikz}
\usetikzlibrary{intersections,backgrounds}
\setitemize{
    label=\usebeamerfont*{itemize item}%
    \usebeamercolor[fg]{itemize item}
    \usebeamertemplate{itemize item}
}
\usepackage{media9}
%\usepackage{pdfpcnotes}
\usepackage{hyperref}
\usepackage{xcolor}
\usepackage[labelformat=empty, justification=centering]{caption}
%\setbeamertemplate{footline}[frame number]{}
%\setbeamertemplate{footline}{}
%\setbeamertemplate{page number in head/foot}{}
\setbeamertemplate{footline}[frame number]
\setbeamertemplate{headline}{}
\setbeamertemplate{navigation symbols}{}
\newcommand{\backupbegin}{
   \newcounter{finalframe}
   \setcounter{finalframe}{\value{framenumber}}
}
\newcommand{\backupend}{
   \setcounter{framenumber}{\value{finalframe}}
}
\title{\textbf{Computational methods for nuclear engineering}}
\subtitle{Workshop on Computational Nuclear Science and Engineering}
\date{2021-07-12}
\institute{IAEA}
\titlegraphic{\includegraphics[width=\textwidth,height=.3\textheight,keepaspectratio]{img/univac-console}}
\author[Touran]{Nick Touran, Ph.D., PE \\ \texttt{ntouran@terrapower.com} }
\begin{document}
%% title frame

\begin{frame}
\titlepage
\end{frame}

\begin{frame}{About me}
    \begin{columns}
    \begin{column}{0.6\textwidth}
    \begin{itemize}
        \item Started programming on an Osborne 1 in childhood
        \item Nuclear engineering Ph.D. from University of Michigan
        \item Been working at TerraPower, LLC since 2009 
        \item Core design, fast reactor physics, software, business
        \item P.E. in State of WA in 2019
        \item Current role: Manager of Digital Engineering
        \item On the side: I run \href{https://whatisnuclear.com}{whatisnuclear.com}
    \end{itemize}
    \end{column}
    \begin{column}{0.4\textwidth}
        \begin{figure}[ht]
        \centering
            \includegraphics[height=0.80\textheight,width=0.9\textwidth,keepaspectratio]{img/nick_and_dog}
            \caption{Me and Waffles, several months ago}
        \end{figure}
    \end{column}
\end{columns}
\end{frame}

\begin{frame}{Technical questions needing answers in nuclear engineering}
        \begin{itemize}
            \item \textbf{Radiation transport} Where are the neutrons and photons? Where
                are they headed? How fast are nuclei splitting? What power are they
                generating, and where?
            \item \textbf{Thermal/hydraulics} How fast must we flow coolant to carry away
                the heat? What are the resulting pressures and vibrations?
            \item \textbf{Fuel performance} What are the mechanical and chemical
                dynamics of fuel system given the nuclear reactions and irradiation?
            \item \textbf{Mechanical} What are the mechanical loads, dynamics, and
                vibrations amongst the fuel assemblies and/or other structures?
            \item \textbf{Plant systems} How do the pumps, pipes, heat exchangers,
                instrumentation, etc. perform in expected and postulated on- and off-normal
                conditions?
        \end{itemize}
\end{frame}

\begin{frame}{Experiments and observations form the initial answers}
\begin{figure}
\begin{tabular}{cccc}
\subfloat{\includegraphics[width=0.2\textwidth,height=0.4\textheight,keepaspectratio]{img/zppr.jpg}} &
\subfloat{\includegraphics[width=0.2\textwidth,height=0.4\textheight,keepaspectratio]{img/are_mockup.jpg}} &
\subfloat{\includegraphics[width=0.2\textwidth,height=0.4\textheight,keepaspectratio]{img/hallam-cr-test-tower.jpg}} &
\subfloat{\includegraphics[width=0.2\textwidth,height=0.4\textheight,keepaspectratio]{img/appr1-core-in-crit-facility.jpg}} \\
\subfloat{\includegraphics[width=0.2\textwidth,height=0.4\textheight,keepaspectratio]{img/hengelo_sg_test_facility}} &
\subfloat{\includegraphics[width=0.2\textwidth,height=0.4\textheight,keepaspectratio]{img/superheat-critical.jpg}}&
\subfloat{\includegraphics[width=0.2\textwidth,height=0.4\textheight,keepaspectratio]{img/fsv_psrv_test.jpg}} &
\subfloat{\includegraphics[width=0.2\textwidth,height=0.4\textheight,keepaspectratio]{img/mtr_1955_hd.6d.017.jpg}}
\end{tabular}
\caption{Some historical nuclear experiments}
\end{figure}
\end{frame}

\begin{frame}{Scale models can be considered analog computers}
\begin{columns}
    \begin{column}{0.5\textwidth}
        \begin{figure}[ht]
        \centering
        \includegraphics[height=0.75\textheight,width=\textwidth,keepaspectratio]{img/milling_yellow_rig}
            \caption{\tiny This rig takes pictures through a special plexiglass analog
            of the HRE vessel}
        \end{figure}
    \end{column}
    \begin{column}{0.5\textwidth}
        \begin{figure}[ht]
        \centering
        \includegraphics[height=0.75\textheight,width=\textwidth,keepaspectratio]{img/milling_yellow}
            \caption{\tiny Flow patterns due to flow double refraction appear in a
            Milling-Yellow solution}
        \end{figure}
    \end{column}
\end{columns}
    \centering \tiny HRE Progress Report July 1958
\end{frame}

\begin{frame}{Analog computer to measure the power coefficient of reactivity}
\begin{columns}
    \begin{column}{0.5\textwidth}
        \begin{figure}[ht]
        \centering
        \includegraphics[height=0.85\textheight,width=\textwidth,keepaspectratio]{img/hallam-analog-computer.png}
        \caption{\small Analog computer at Hallam}
        \end{figure}
    \end{column}
    \begin{column}{0.5\textwidth}
        \begin{figure}[ht]
        \centering
        \includegraphics[height=0.85\textheight,width=\textwidth,keepaspectratio]{img/hallam-kinetic-circuit-diagram}
        \caption{\small The schematics}
        \end{figure}
    \end{column}
\end{columns}
    \centering \tiny \url{https://babel.hathitrust.org/cgi/pt?id=mdp.39015095040682&view=1up&seq=32}
\end{frame}

\begin{frame}{Nuclear engineers were some of the first major users of digital computers}
        \begin{figure}[ht]
        \centering
            \includegraphics[height=0.90\textheight,width=\textwidth,keepaspectratio]{img/ebr2-univac.png}
    \caption{\small Digital computers were used in 1955 for reactor design
            (\url{https://babel.hathitrust.org/cgi/pt?id=mdp.39015003993188&view=1up&seq=1})}
        \end{figure}
\end{frame}

\begin{frame}{There were dozens upon dozens of nuclear codes in 1964!}
        \begin{figure}[ht]
        \centering
            \includegraphics[height=0.75\textheight,width=\textwidth,keepaspectratio]{img/1964-diffusion-codes}
    \caption{\small From Elizabeth Cuthill - Digital Computers in Nuclear Reactor Design, 1964
            (\url{https://doi.org/10.1016/S0065-2458(08)60356-3})}
        \end{figure}
\end{frame}



\begin{frame}{Today, computation is used for a variety of aims}
\begin{columns}
    \begin{column}{0.7\textwidth}
        \begin{itemize}
            \item General engineering/design without the challenges of classical analysis
            \item Run many perturbed cases to seek a multi-objective optimum
            \begin{itemize}
                \item in trade studies backing design decisions
                \item in preparation for targeted experiments 
                \item to estimate complex system-level sensitivities
            \end{itemize}
            \item Explore a new space in reactor design
            \item Seek insight into some complexity with ultra-high fidelity
        \end{itemize}
    \end{column}
    \begin{column}{0.3\textwidth}
        \begin{figure}[ht]
        \centering
            \includegraphics[height=0.80\textheight,width=0.9\textwidth,keepaspectratio]{img/computer-user.png}
        \end{figure}
    \end{column}
\end{columns}
\end{frame}

\begin{frame}{Basic approach to deterministic computational nuclear engineering}
    \begin{columns}
    \begin{column}{0.4\textwidth}
    \begin{itemize}
        \item Write a balance statement
        \item Specify boundary conditions
        \item Make assumptions to simplify as necessary
        \item Discretize the solution space considering geometry and material properties
        \item Cast balance statement into matrix form
        \item Solve matrix equation numerically
        \item Provide results in intuitive and useful way
    \end{itemize}
    \end{column}
    \begin{column}{0.4\textwidth}
        \begin{figure}[ht]
        \centering
            \includegraphics[height=0.80\textheight,width=0.9\textwidth,keepaspectratio]{img/q2.png}
        \end{figure}
    \end{column}
\end{columns}

\end{frame}

\begin{frame}[plain,c]
	\begin{center}
	\Huge Neutronics
	\end{center}
\end{frame}

\begin{frame}{The Neutron Transport Equation is a balance statement}
    \begin{center}
        \begin{figure}[ht]
        \centering
        \includegraphics[height=0.80\textheight,width=0.9\textwidth,keepaspectratio]{img/neutron-transport-eqn}
            \caption{\tiny The neutron transport equation}
        \end{figure}
    \end{center}
    Given a solution to this, you get nuclear reaction rates, which give:
    \begin{itemize}
        \item Gamma source distribution (leads to another transport solve)
        \item Power distribution (cooling, distortions)
        \item Dose rate distribution (material damage, detector signal)
        \item Dynamics (operations, safety)
    \end{itemize}

\end{frame}

\begin{frame}{The domain: a nuclear core}
\begin{columns}
    \begin{column}{0.5\textwidth}
        \begin{figure}[ht]
        \centering
        \includegraphics[height=0.80\textheight,width=\textwidth,keepaspectratio]{img/hallam_core}
            \caption{\small The Hallam core}
        \end{figure}
    \end{column}
    \begin{column}{0.5\textwidth}
        \begin{figure}[ht]
        \centering
        \includegraphics[height=0.85\textheight,width=\textwidth,keepaspectratio]{img/hallam_cell}
        \caption{\small A large 'unit cell'}
        \end{figure}
    \end{column}
\end{columns}
\end{frame}

\begin{frame}{Spatial discretization}
\begin{columns}
    \begin{column}{0.5\textwidth}
        \begin{figure}[ht]
        \centering
        \includegraphics[height=0.85\textheight,width=\textwidth,keepaspectratio]{img/dif3d-triz.png}
        \caption{\small A Triangular-Z volume element}
        \end{figure}
    \end{column}
    \begin{column}{0.5\textwidth}
        \begin{figure}[ht]
        \centering
        \includegraphics[height=0.85\textheight,width=\textwidth,keepaspectratio]{img/dif3d-trimesh.png}
        \caption{\small A 120° domain}
        \end{figure}
    \end{column}
\end{columns}
    \centering \tiny From \href{https://digital.library.unt.edu/ark:/67531/metadc283553/}{DIF3D: A Code to Solve One-, Two-, and Three-Dimensional
    Finite-Difference Diffusion Theory
    Problems}
\end{frame}

\begin{frame}{Some approximations to the transport equation}

Steady state, multi-group diffusion:
\begin{equation}
\left(D_{g}+\Sigma_{g}\right)\phi_{g}-\sum_{g^{\prime}\ne g}T_{gg^{\prime}}\phi_{g}=\frac{1}{\lambda}\chi_{g}\sum_{g^{\prime}=1}^{G}F_{g^{\prime}}\phi_{g^{\prime}},\label{eq:group-balance-orig}
\end{equation}


\begin{equation}
M=\left[\begin{array}{cccc}
\left[A_{1}\right] &  &  & \left[0\right]\\
 & \left[A_{2}\right]\\
 &  & \ddots\\
\left[0\right] &  &  & \left[A_{G}\right]
\end{array}\right]-\left[\begin{array}{ccccc}
[0] & \left[T_{12}\right] & \cdots & \cdots & \left[T_{1G}\right]\\
\left[T_{21}\right] & [0] &  &  & \vdots\\
\vdots &  & \ddots &  & \vdots\\
\vdots &  &  & \ddots & \left[T_{G-1,G}\right]\\
\left[T_{G1}\right] & \left[T_{G2}\right] & \cdots & \left[T_{G,G-1}\right] & [0]
\end{array}\right],
\end{equation}

Eigenvalue equation:
\begin{equation}
M\phi=\frac{1}{\lambda}B\phi,\label{eq:generalizedEigenproblem}
\end{equation}

\end{frame}


\begin{frame}[t]{Building the matrices}
    \begin{columns}[onlytextwidth, t]
    \column{0.5\textwidth}
    \begin{equation}
    \resizebox{0.8 \textwidth}{!}
    {
        $
        A_{g}=\left[\begin{array}{ccccccc}
        \ddots & \ddots &  & \ddots\\
         & \left[A_{gx}^{J-1,k}\right] & \left[A_{gy}^{J,k}\right] &  & \left[A_{gz}^{J-1,k+1}\right]\\
         &  & \left[A_{gx}^{J,k}\right] & \left[0\right] &  & \left[A_{gz}^{J,k+1}\right]\\
         & \text{} &  & \left[A_{gx}^{1,k+1}\right] & \left[A_{gy}^{2,k+1}\right] &  & \ddots\\
         &  &  &  & \ddots & \ddots
        \end{array}\right],
        $
    }
    \end{equation}

    \begin{equation}
    \resizebox{0.8 \textwidth}{!}
    {
        $
        \left[A_{gx}^{j,k}\right]=\left[\begin{array}{ccccccc}
        b_{1} & -a_{2}^{x}\\
        -a_{2}^{x} & b_{2} & -a_{3}^{x}\\
         & \ddots & \ddots & \ddots\\
         &  & \ddots & \ddots & \ddots\\
         &  &  & \ddots & \ddots & \ddots\\
         &  &  &  & -a_{I-1}^{x} & b_{I-1} & -a_{I}^{x}\\
         &  &  &  &  & -a_{I}^{x} & b_{I}
        \end{array}\right]_{ikg},
        $
    }
    \end{equation}
        \column{0.5\textwidth}
            \centering
            \begin{figure}[ht]
                \includegraphics[height=0.65\textheight,width=\textwidth,keepaspectratio]{img/fd_matrix.jpg}
            \caption{\small A 7-banded diffusion matrix for one group in 1/3-core 3D triangular geometry}
            \end{figure}
    \end{columns}
\end{frame}

\begin{frame}{We have high fidelity deterministic codes these days}
\begin{columns}
    \begin{column}{0.5\textwidth}
        \begin{itemize}
            \item Unstructured mesh
            \item Detailed treatment of scattering
            \item Requires cross section averaging lattice calculations
            \item Powerful, iterative matrix solvers (e.g. LAPACK, PETSc, Trilinos)
        \end{itemize}
    \end{column}
    \begin{column}{0.5\textwidth}
        \begin{figure}[ht]
        \centering
            \includegraphics[height=0.55\textheight,width=\textwidth,keepaspectratio]{img/rattlesnake-atr-mesh}
        \caption{\tiny A fine mesh of the ATR from
            \url{https://rattlesnake.inl.gov/SitePages/Home.aspx}}
        \end{figure}
    \end{column}
\end{columns}
\end{frame}

\begin{frame}{Nuclear data evaluations underly transport models}
        \begin{itemize}
            \item Many measurements are made
            \item Nuclear models run to interpolate between data
            \item Complex statistics and Bayesian inference to get 'best' fit
        \end{itemize}
\begin{columns}
    \begin{column}{0.5\textwidth}
        \begin{figure}[ht]
        \centering
        \includegraphics[height=0.55\textheight,width=\textwidth,keepaspectratio]{img/pu239-endf8}
        \caption{\small The evaluated data}
        \end{figure}
    \end{column}
    \begin{column}{0.5\textwidth}
        \begin{figure}[ht]
        \centering
            \includegraphics[height=0.55\textheight,width=\textwidth,keepaspectratio]{img/pu239-exfor}
        \caption{\small A small selection of experimental data}
        \end{figure}
    \end{column}
\end{columns}
    \centering \tiny \url{https://www.oecd-nea.org/janisweb}
\end{frame}

\begin{frame}{Averaging cross sections across regions and energies is a complex art}
\begin{columns}
    \begin{column}{0.5\textwidth}
        \begin{itemize}
            \item Strong resonances from one nuclear depress flux, impact averaging of
                others
            \item Spatial details vs. spectral details matter differently depending on
                composition
            \item Nearby assemblies can impact the averaging as well
            \item Many lattice codes are specialized for a subset of reactor designs
        \end{itemize}
    \end{column}
    \begin{column}{0.5\textwidth}
        \begin{figure}[ht]
        \centering
            \includegraphics[height=0.55\textheight,width=\textwidth,keepaspectratio]{img/lwr-sfr-assem-neutrons}
        \caption{\tiny What slow neutrons see vs. fast neutrons}
        \end{figure}
    \end{column}
\end{columns}
\end{frame}


\begin{frame}{Monte Carlo: explicitly track particle histories}
\begin{columns}
    \begin{column}{0.5\textwidth}
        \begin{itemize}
            \item No geometry approximations
            \item No angular approximations
            \item No multi-group approximations
            \item No cross section averaging
            \item \textbf{But...} lots and lots of histories needed
            \item Big computers
            \item Lots of time
        \end{itemize}
    \end{column}
    \begin{column}{0.5\textwidth}
        \begin{figure}[ht]
        \centering
            \includegraphics[height=0.75\textheight,width=\textwidth,keepaspectratio]{img/monte_carlo_casino}
            \caption{\tiny The Monte Carlo Casino (CC-BY-2.0 Lolipouette)} 
        \end{figure}
    \end{column}
\end{columns}
\end{frame}

\begin{frame}{Monte Carlo: basics}
        \begin{itemize}
            \item Roll dice: sample distance to next interaction
            \item Sample from cross section data to choose reaction
            \item Sample from cross section data to choose outgoing energy and direction
            \item Repeat billions of times, tallying up track length \& reaction rates
        \end{itemize}
\begin{columns}
    \begin{column}{0.5\textwidth}
        \begin{figure}[ht]
        \centering
            \includegraphics[height=0.55\textheight,width=\textwidth,keepaspectratio]{img/lwr-xs-sq-line}
            \caption{\tiny Look at the macroscopic cross sections at the particle's current energy} 
        \end{figure}
    \end{column}
    \begin{column}{0.5\textwidth}
        \begin{figure}[ht]
        \centering
            \includegraphics[height=0.55\textheight,width=\textwidth,keepaspectratio]{img/mc-reaction-choice}
            \caption{\tiny A discrete CDF for material choice} 
        \end{figure}
    \end{column}
\end{columns}
\end{frame}

\begin{frame}{Transmutation and decay}

\begin{equation}
\frac{\partial N_{i}}{\partial t}=-N_{i}\left(\sum_{g=1}^{G}\sigma_{a,g}^{i}\phi_{g}+\lambda_{i}\right)+\sum_{j=1}^{I}N_{j}\left(\sum_{g=1}^{G}\left(\gamma_{g}^{j\rightarrow i}\phi_{g}\right)+\lambda_{j\rightarrow i}\right),\label{eq:transmutation_rates}
\end{equation}

    \begin{center}

    \includegraphics[height=0.6\textheight,width=\textwidth,keepaspectratio]{img/transmutation}\\
    Solution: $N(t) = e^{A t} N(0)$
    \end{center}
\end{frame}

\begin{frame}{A warning about limitations}
        \begin{itemize}
            \item Underlying data have uncertainties! Keep track of them.
            \item The value of drilling down to 0.5 pcm without a specialized application
                in mind is dubious when there's 200-500 pcm data uncertainty
        \end{itemize}
\begin{columns}
    \begin{column}{0.5\textwidth}
        \begin{figure}[ht]
        \centering
            \includegraphics[height=0.55\textheight,width=\textwidth,keepaspectratio]{img/na23-elastic}
        \end{figure}
    \end{column}
    \begin{column}{0.5\textwidth}
        \begin{figure}[ht]
        \centering
            \includegraphics[height=0.55\textheight,width=\textwidth,keepaspectratio]{img/u238-inelastic}
        \end{figure}
    \end{column}
\end{columns}
    \centering \tiny \url{https://www.oecd-nea.org/janisweb}
\end{frame}


\begin{frame}[plain,c]
	\begin{center}
	\Huge Fluid flow and heat transfer
	\end{center}
\end{frame}

\begin{frame}{Fluid flow and heat transfer}
\begin{columns}
    \begin{column}{0.5\textwidth}
        \begin{itemize}
            \item Fundamental to safety performance: need to keep components at safe
                temperatures and pressures
            \item Important to economics: attempt to safely get more power out of same
                equipment
            \item Often strongly coupled to neutronics
        \end{itemize}
    \end{column}
    \begin{column}{0.5\textwidth}
        \begin{figure}[ht]
        \centering
            \includegraphics[height=0.75\textheight,width=\textwidth,keepaspectratio]{img/failed-ebr2-fuel}
            \caption{\tiny Some (intentionally) overheated fuel pins \url{https://babel.hathitrust.org/cgi/pt?id=mdp.39015078511006&view=1up&seq=45}}
        \end{figure}
    \end{column}
    \end{columns}
\end{frame}

\begin{frame}{Important couplings between thermal/hydraulics and neutronics}
\begin{columns}
    \begin{column}{0.5\textwidth}
        \begin{itemize}
            \item Macroscopic nuclear cross sections are temperature dependent (thermal
                expansion, soluble boron)
            \item Microscopic nuclear cross sections are temperature dependent (Doppler effect)
            \item Flow-induced pressure can distort the core, causing changes in
                macroscopic cross section
            \item In fluid fuel reactor or severe accidents, flow moves the fuel and
                delayed neutron precursors
        \end{itemize}
    \end{column}
    \begin{column}{0.5\textwidth}
        \begin{figure}[ht]
        \centering
            \includegraphics[height=0.75\textheight,width=\textwidth,keepaspectratio]{img/mtc}
            \caption{\tiny The Moderator Temperature Coeff in a PWR (Joseph M. Farley
            FSAR Fig. 4.3-30) \url{https://www.nrc.gov/docs/ML0818/ML081820242.pdf}}
        \end{figure}
    \end{column}
    \end{columns}
\end{frame}

\begin{frame}{Thermal/Hydraulics Fundamentals}
\begin{columns}
    \begin{column}{0.7\textwidth}
        Approach is familiar: discretize spatially, make assumptions and simplifications, cast to matrix looking up material
        properties, and solve\\~\

        Governing equations: Navier-Stokes
        \begin{itemize}
            \item Conservation of mass 
            \item Conservation of energy
            \item Conservation of momentum\\~\
        \end{itemize}

        These, plus complex fluid property lookup tables (e.g. steam tables) make T/H
        calculations particularly conducive to computerization

    \end{column}
    \begin{column}{0.3\textwidth}
        \begin{figure}[ht]
        \centering
            \includegraphics[height=0.75\textheight,width=\textwidth,keepaspectratio]{img/navier-stokes}
            \caption{\tiny \href{https://www.grc.nasa.gov/www/k-12/airplane/nseqs.html}{The conservation equations}}
        \end{figure}
    \end{column}
    \end{columns}
\end{frame}


\begin{frame}{Hot channel T/H codes}
\begin{columns}
    \begin{column}{0.7\textwidth}
            \begin{itemize}
                \item 1-D fluid flow
                \item Hot channel factors and correlations for peaking, turbulent mixing,
                    heat transfer, friction factor, and phase slip
                \item Single phase (liquid metal, molten salt, or gas) or two phase (water)
                \item 1- or 2-D conduction models into fuel and structures
                \item Full-core transient calculations trivial
            \end{itemize}
    \end{column}
    \begin{column}{0.3\textwidth}
        \begin{figure}[ht]
        \centering
            \includegraphics[height=0.75\textheight,width=\textwidth,keepaspectratio]{img/th-channel-discret}
            \caption{\tiny 1-D discretization of a single channel \url{https://deepblue.lib.umich.edu/handle/2027.42/89079}}
        \end{figure}
    \end{column}
\end{columns}
\end{frame}

\begin{frame}{Subchannel T/H codes}
\begin{columns}
    \begin{column}{0.7\textwidth}
        \begin{itemize}
            \item Typically still 1-D fluid flow
            \item Communicating subchannels (mixing, conduction)
            \item Intra-assembly heat transfer, often 3-D conduction to fuel
            \item Less reliant on hot channel factors, but still need
                correlations
            \item Single phase or two phase 
            \item Full-core static calculations routine, transients doable
            \item Requires pin-level power distributions
        \end{itemize}
    \end{column}
    \begin{column}{0.3\textwidth}
        \begin{figure}[ht]
        \centering
            \includegraphics[height=0.55\textheight,width=\textwidth,keepaspectratio]{img/cortran-subchannels}
            \caption{\tiny Subchannels from CORTRAN \url{https://www.osti.gov/servlets/purl/6881496}}
        \end{figure}
    \end{column}
\end{columns}
\end{frame}

\begin{frame}{Typical subchannel code physics}
\begin{columns}
    \begin{column}{0.6\textwidth}
        \begin{itemize}
            \item Steady state and transient solvers
            \item Single and two-phase flow models with non-uniform channel friction,
                subcooled voids, etc.
            \item Turbulent mixing 
            \item Crossflow
            \item Equation of state for various coolants
            \item Grid spacer, wire wrap, blocked crossflow
            \item Axial and radial heat conduction into fuel
            \item Axial and radial heat conduction in fluid
        \end{itemize}
    \end{column}
    \begin{column}{0.4\textwidth}
        \begin{figure}[ht]
        \centering
            \includegraphics[height=\textheight,width=0.8\textwidth,keepaspectratio]{img/CTF_mpact.png}
            \caption{\tiny Subchannel results in Watts Bar 1 from CTF at ORNL \url{https://www.ornl.gov/division/rnsd/projects/ctf}}
        \end{figure}
    \end{column}
\end{columns}
\end{frame}


\begin{frame}{Computational Fluid Dynamics}

\begin{columns}
    \begin{column}{0.7\textwidth}
    \begin{itemize}
        \item High-resolution mesh 
        \item Several methods with different turbulence models
        \item Multi-phase flow still under development
        \item Used to verify use of correlations in faster subchannel codes
        \item Used for thermal striping, stratification, etc.
        \item Used during trade studies
        \item Useful in exploring and understanding new design space lacking in
            experiments
        \item Helps design the most impactful experiments
    \end{itemize}
    \end{column}
    \begin{column}{0.3\textwidth}
        \begin{figure}[ht]
        \centering
            \includegraphics[height=0.55\textheight,width=\textwidth,keepaspectratio]{img/plenum-mixing}
            \caption{\tiny Obabko, 2015
            \url{https://www.mcs.anl.gov/papers/P5386-0715.pdf}}
        \end{figure}
    \end{column}
\end{columns}
\end{frame}

\begin{frame}{Trivial T/H solution for 1-D single-phase flow}
\begin{columns}
    \begin{column}{0.5\textwidth}
        \begin{figure}[ht]
        \centering
            \includegraphics[height=0.55\textheight,width=\textwidth,keepaspectratio]{img/th-code-massflow}
            \caption{\tiny Compute the required mass flow rate given the assembly power } 
        \end{figure}
    \end{column}
    \begin{column}{0.5\textwidth}
        \begin{figure}[ht]
        \centering
            \includegraphics[height=0.55\textheight,width=\textwidth,keepaspectratio]{img/th-code-1d}
            \caption{\tiny Given mass flow rate, compute the axial coolant temperature
            distribution} 
        \end{figure}
    \end{column}
\end{columns}
\end{frame}

\begin{frame}[plain,c]
	\begin{center}
	\Huge Core Mechanical
	\end{center}
\end{frame}

\begin{frame}{Core Mechanical Analysis}
    Understanding how the fuel assemblies interact with the support structures
    \begin{columns}
    \begin{column}{0.5\textwidth}
        \begin{itemize}
            \item Seismic analysis important for safety
            \item Vibrations can cause premature fuel failure
            \item Fuel management loads needed for reload operations
            \item Core distortions can have large impacts on the safety case in some
                reactors
        \end{itemize}
    \end{column}
    \begin{column}{0.5\textwidth}
        \begin{figure}[ht]
        \centering
            \includegraphics[height=0.55\textheight,width=\textwidth,keepaspectratio]{img/nubow-feedback}
            \caption{\tiny Core radial expansion's impact on FFTF reactivity
            \url{https://www.osti.gov/servlets/purl/5236450} }
        \end{figure}
    \end{column}
\end{columns}
\end{frame}

\begin{frame}{The Finite Element Method}
    \begin{columns}
    \begin{column}{0.5\textwidth}
        \begin{itemize}
            \item Works well with highly complex shapes
            \item Governing equations: $F = ma$ and  $\sum{F} = 0$ 
            \item Matrix size is proportional to the mesh resolution
        \end{itemize}
    \end{column}
    \begin{column}{0.5\textwidth}
        \begin{figure}[ht]
        \centering
            \includegraphics[height=0.85\textheight,width=\textwidth,keepaspectratio]{img/bowing}
            \caption{\tiny A model of a core subjected to an acceleration
            (from TerraPower LLC) }
        \end{figure}
    \end{column}
\end{columns}
\end{frame}

\begin{frame}[plain,c]
	\begin{center}
	\Huge Fuel Performance
	\end{center}
\end{frame}

\begin{frame}{Fuel Performance}
    Reliable prediction of fuel behavior in normal and off-normal conditions.
    \begin{columns}
    \begin{column}{0.5\textwidth}
        \begin{itemize}
            \item Important for safety and operations-related issues like fission product retention
            \item Tied to fuel thermal and burnup limits and therefore economics and
                sustainability
            \item Typically computes temperature, stress, strain, fission gas pressure,
                fission gas release, cladding corrosion, and cladding erosion within a
                fuel pin/clad system
            \item Arguably should include fuel salt chemistry models for fluid fuel
        \end{itemize}
    \end{column}
    \begin{column}{0.5\textwidth}
        \begin{figure}[ht]
        \centering
            \includegraphics[height=0.5\textheight,width=0.5\textwidth,keepaspectratio]{img/fuelcycle4-fab}
            \caption{\tiny Fresh fuel pellets}
        \end{figure}
        \begin{figure}[ht]
        \centering
            \includegraphics[height=0.5\textheight,width=0.5\textwidth,keepaspectratio]{img/bad-fuel}
            \caption{\tiny Irradiated fuel slugs (from the olden days)
            \url{https://www.osti.gov/servlets/purl/4300801}} 
        \end{figure}
    \end{column}
\end{columns}
\end{frame}

\begin{frame}{Fuel Performance computational approach}
    \begin{columns}
    \begin{column}{0.6\textwidth}
        \begin{itemize}
            \item Inherently dependent on difficult-to-measure correlations e.g. for burnup-dependent
                properties
            \item Beware overfitting the experimental data
            \item Historical codes use several hundred mesh points and run in seconds on
                a modern PC
            \item Solves solid mechanics statics/dynamics equations including irradiation
                strain models using finite difference or finite element methods
            \item Several modern codes make fine-mesh problems and can leverage supercomputers
        \end{itemize}
    \end{column}
    \begin{column}{0.4\textwidth}
        \begin{figure}[ht]
        \centering
            \includegraphics[height=0.55\textheight,width=\textwidth,keepaspectratio]{img/fraptran}
            \caption{\tiny A typical FRAPTRAN computational mesh
            \url{https://www.nrc.gov/docs/ML0125/ML012530260.pdf}} 
        \end{figure}
    \end{column}
\end{columns}
\end{frame}
\begin{frame}{Fuel Performance computational approach}

    \begin{columns}
    \begin{column}{0.5\textwidth}
        Couplings
        \begin{itemize}
            \item Fission gas release and axial/radial strain can seriously impact
                neutronics in many cases
            \item Coupling with thermal/hydraulics provides fuel temperature limits
            \item Clad strain can impact coolant flow
            \item For fluid fuel, coupling with neutronics is fundamental
        \end{itemize}
    \end{column}
    \begin{column}{0.5\textwidth}
        \begin{figure}[ht]
        \centering
            \includegraphics[height=0.55\textheight,width=\textwidth,keepaspectratio]{img/fuel-perf-data}
            \caption{\tiny Fuel performance correlations from FRAPTRAN
            \url{https://fast.labworks.org/file-download/download/public/108} }
        \end{figure}
    \end{column}
\end{columns}
\end{frame}



\begin{frame}[plain,c]
	\begin{center}
	\Huge Systems codes
	\end{center}
\end{frame}

\begin{frame}{Systems codes}
\begin{columns}
    \begin{column}{0.5\textwidth}
    \begin{itemize}
        \item Simulate the entire plant in various operational and off-normal conditions
        \item Front and center in licensing
        \item Includes representations of major equipment in balance of plant
            (sometimes in lower fidelity than the core)
    \end{itemize}
    \end{column}
    \begin{column}{0.5\textwidth}
        \begin{figure}[ht]
        \centering
            \includegraphics[height=\textheight,width=0.9\textwidth,keepaspectratio]{img/pwr-anim-still}
            \caption{\tiny 
            \href{https://commons.wikimedia.org/w/index.php?title=File\%3APWR_nuclear_power_plant_animation.webm}{Plant
            systems in a PWR}}
        \end{figure}
    \end{column}
\end{columns}
\end{frame}


\begin{frame}{Simple discretization lumped-parameter models}
\begin{columns}
    \begin{column}{0.5\textwidth}
    \begin{itemize}
        \item Coolant property lookup tables
        \item Includes or is heavily informed by subchannel code and fuel performance code
        \item Includes point or spatial kinetics nuclear models
        \item Simple models are conducive to running many hundreds of cases to support the
            licening basis
        \item Experimental validation sometimes easier when simple and conservative
        \item Handle varying degrees of severe accident
    \end{itemize}
    \end{column}
    \begin{column}{0.5\textwidth}
        \begin{figure}[ht]
        \centering
            \includegraphics[height=0.55\textheight,width=\textwidth,keepaspectratio]{img/th-discretization}
            \caption{\tiny Discretization in TOPAZ
            \url{https://www.osti.gov/servlets/purl/5437631}}
        \end{figure}
    \end{column}
\end{columns}
\end{frame}
\begin{frame}[plain,c]
	\begin{center}
	\Huge Information Management
	\end{center}
\end{frame}

\begin{frame}{Multiphysics analysis suites}
\begin{columns}
    \begin{column}{0.5\textwidth}
    \begin{itemize}
        \item Automate time-consuming analysis workflows 
        \item Capture important coupled physics effects for complex analyses
        \item Write full analysis methodologies as code (with rigorous QA)
        \item Leverage multiobjective optimization systems
        \item Machine learning, AI, digital twin, etc.
    \end{itemize}
    \end{column}
    \begin{column}{0.5\textwidth}
        \begin{figure}[ht]
        \centering
            \includegraphics[height=0.85\textheight,width=\textwidth,keepaspectratio]{img/automation}
            \caption{\tiny A person (or supercomputer) running full-scope calculations
            rapidly}
        \end{figure}
    \end{column}
\end{columns}
\end{frame}

\begin{frame}{Project and Asset Information Management}
\begin{columns}
    \begin{column}{0.5\textwidth}
    \begin{itemize}
        \item Indirect costs are now ~50\% of new nuclear costs!
        \item Inefficiencies in high-level work process are ripe for disruption
        \item Computers as information management and process facilitators
        \item See:
            \href{https://www.iaea.org/publications/8438/information-technology-for-nuclear-power-plant-configuration-management}{IAEA-TECDOC-1651},
            \href{https://www.epri.com/research/products/3002003126}{EPRI 3002003126}, and
            \href{http://cmbg.org}{cmbg.org}.
    \end{itemize}
    \end{column}
    \begin{column}{0.5\textwidth}
        \begin{figure}[ht]
        \centering
            \includegraphics[height=0.85\textheight,width=\textwidth,keepaspectratio]{img/data_hub}
            \caption{\tiny A data hub for nuclear enterprise
            \url{https://www-pub.iaea.org/MTCD/Publications/PDF/te_1651_web.pdf}}
        \end{figure}
    \end{column}
\end{columns}
\end{frame}

\begin{frame}[plain,c]{Conclusions}
    \begin{itemize}
        \item Computers have been and will remain incredible tools for nuclear engineers
        \item Each area has still-relevant methods (and in some cases codes!) from the 1980s and extraordinary
            high-fidelity modern codes for specialized applications
        \item This only barely scratches the surface; dozens of publications per month
        \item The magnitude of mechanical, civil/structural, electrical, and operations
            engineering at nuclear plants hasn't even been mentioned
        \item There are parts of the nuclear enterprise that have not yet fully leveraged
            computers
    \end{itemize}
	\begin{center}
	\Huge Discussion!\\
    \includegraphics[width=0.1\textwidth,keepaspectratio]{img/reactor.png}\\
    \normalsize Or email me: ntouran@terrapower.com
	\end{center}
\end{frame}
\appendix
\backupbegin




\backupend

\end{document}
